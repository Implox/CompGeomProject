\documentclass[letterpaper,oneside,12pt]{article}

\usepackage{natbib}

\begin{document}
  \nocite{*}
  \pagestyle{plain}
  \begin{center}
    {\LARGE Kinetic Data Structures Project Proposal} \\ 
    {\Large Alex Flores, Parker Given, Emmit Tran } \\
    {\large \today } \\
  \end{center}

  \section*{Overview}

  \noindent When modeling the interactions of continuously-moving objects in a physical
  scene it is useful to keep track of certain attributes such as the closest pair of
  objects in the scene, the convex hull of all objects, the minimum spanning tree of
  the objects, and so on. Data structures for computing and maintaining such attributes 
  are well known, however they are generally designed with the assumption that the 
  configuration of each item in the scene is immutable. Using these structures 
  for representing scenes which change over time often results in inefficient computations. 
  To handle scenes which change over time, a specialized category of structures known as
  \textit{kinetc data structures}~(KDEs) have been developed.
  
  \noindent The purpose of this project is to conduct a survey of common kinetic data
  structures and to develop software which uses visualizations to demonstrate and describe
  the structures and their inner workings.

  \section*{Deliverables}
  A graphical software which calculates and displays common attributes (e.g.: closest pair, 
  convex hull, etc.) in a continuously-changing system using kinetic data structures, which
  also allows a user to see a visualization of the underlying data structure and the intermediate
  steps involved in computing the attributes.

  \renewcommand\refname{Tentative Bibliography}
  \bibliography{citations}
  \bibliographystyle{plain}

\end{document}